/* DO NOT MODIFY THIS FILE IN ANY WAY!!!  SEE WARNING AT END OF FILE. */
/* Lines for Files to Delete from the .ZIP File: input box: */
Specify the files you want to delete.  Use of wildcards    
is fine, so you can type *.COM *.EXE to delete all the   
files with those extensions from the .ZIP file.  Or you  
can type AUTOEXEC.BAT CONFIG.SYS to delete just those  
two files.  Or, if you want a checkbox from which you     
can choose whatever individual files you want, click   
on CANCEL here.  
Just be aware that many DOS programs (like zippers)      
can't handle command lines longer than 127 characters.    
So don't specify too many individual filenames!   
/* Lines for .ZIP File to Create: input box: */
Type the full drive:\pathname\filename of the file you   
want to create.  For example, D:\ZIPENGIN.ZIP or     
C:\UTIL\ZIPFILES\TESTING.ZIP      
Your default directory name is already filled in for 
you if you have used the Edit-the-Config routine to   
specify one.  If you don't want to use that directory  
name this time, then just backspace over it and type    
a new name.  Or if you do, then just add a filename    
to the end of it.
/* Lines for Target Directory Name: input box: */
   Type the full drive:\pathname of the directory into which you want to     
place the extracted files.  Like C:\UTIL or A:\ or B:\DOWN\NEW       
   If there's already a directory name there, and you don't want to use that      
one, just backspace or edit it.
   If the directory you want to unzip into does not exist, Zip Engine will   
create it as long as its parent exists.  For example, if C:\DOWN\NEW does       
not exist but C:\DOWN does, then Zip Engine will create NEW if you say so.       
   If you want to create a directory with the same name as the file you're     
unzipping, use $ as the directory name.  Like if your .ZIP file is THIS.ZIP      
then type C:\UNZIP\$ to have Zip Engine create C:\UNZIP\THIS        
/* Lines for Directory of Files to Zip: input box: */
Type the full drive:\pathname of the directory which contains the files    
you want to zip.  For example, C:\PRODIGY\PRO-UTIL or C:\OS2          
or D:\ or whatever. 
Or if you're not adding any files to the .ZIP file you're working with,  
choose CANCEL now to continue without specifying any files to add, and       
we'll add '-x*.*' to the command line instead, so that the Zip program     
won't just add all the files from the current directory as it would by  
default.
/* Lines for Files to Zip: input box: */
Now specify the files you want to zip from that directory.   
For example, *.* for all the files, or *.EXE for all the 
.EXE files, or TEXT*.*  *.EXE for all the .EXE files   
and all the files that start with TEXT, or AUTOEXEC.BAT      
CONFIG.SYS to do just those two files.  Or, if you want    
a checkbox from which you can choose whatever individual      
files you want, click on CANCEL here.   
Just be aware that many DOS programs (like zippers) can      
not handle command lines longer than 127 characters.  So     
don't specify too many individual filenames!  
/* Lines for Careful! message box: */
This command may not be successful.  DOS programs can only
process the first 127 characters of any command, and OS/2 passes
some extra characters to the DOS session.  So watch the number
in the lower right corner of the 'Watch the Command Line Being
Built' window (which is the length of the command you're building)
and if your command does not work, then you'll have an idea of
what the limit is on your system so that next time, you'll know
not to exceed that limit.  If you often run into this problem,
then you should consider getting an OS/2 zipper to replace your
current one.
/* Lines for Finished building command line.... message box: */
The Command Line is complete now.  It is shown in the dialog box below, 
and you may edit it if desired (especially in cases where you have chosen 
default options in the Edit-the-Config routine and you want to get rid of
one of them for this particular operation).  If you edit it and change
your mind, clicking on CANCEL will allow you to execute the command 
as it was before you edited it, or alternately abort the entire opera-
tion.  Click on OK in the dialog box below when you're satisfied with the 
command and ready to execute it.  Note that if you have a long command it 
is split into sections.  Each can be no longer than 80 characters.  And
be very careful not to add or remove spaces at the ends of each section!
/* Lines for Cancel? message box: */
Do you want to execute the command as it was before you edited
it, or do you want to abort the operation and return to the
main menu?  OK to execute, or CANCEL for main menu.
/* Lines for Sorry! message box under the -& switch in Zopt: */
You cannot use the 'paths and subdirs' switches
with the -&s switch.  That's ok, because the -&s
takes care of everything the 'paths and subdirs'
switches do.  Dropping the 'paths and subdirs'
switches now.
/* Lines for Files to Exclude from the .ZIP File: input box: */
Specify the files you want to exclude.  Use of wild-   
cards (* or ?) is fine.  For example, you might type   
*.EXE  *.COM  *.BAT  to exclude all the files with    
those extensions.  Or say AUTOEXEC.BAT CONFIG.SYS           
to exclude just those two files.  Or, if you want a  
checkbox from which you can choose whatever indiv-      
idual files you want, click on CANCEL here.   
Just be aware that many DOS programs (like zippers)     
can't handle command lines longer than 127 characters.      
So don't specify too many individual filenames!   
/* Lines for Files to Exclude from Extraction: input box: */
Specify the files you want to exclude.  Use of wild-   
cards (* or ?) is fine.  For example, you might type   
*.EXE  *.COM  *.BAT  to exclude all the files with    
those extensions.  Or say AUTOEXEC.BAT  CONFIG.SYS          
to exclude just those two files.  Or, if you want a  
checkbox from which you can choose whatever indiv-      
idual files you want, click on CANCEL here.   
Just be aware that many DOS programs (like unzippers)       
can't handle command lines longer than 127 charac-    
ters.  So don't specify too many individual filenames!  
/* Lines for Really? message box: */
Are you sure you want to relabel the entire target drive
with the volume label that's inside the .ZIP file, if any?
Click on NO to drop the -$ switch.
/* Lines for Files to Extract from the .ZIP File: input box: */
Specify the files you want to extract.  Use of wildcards   
is fine, so you can type *.COM *.EXE to extract all the   
files with those extensions from the .ZIP file.  Or you  
can type AUTOEXEC.BAT CONFIG.SYS to extract just those          
two files.  Or, if you want a checkbox from which you     
can choose whatever individual files you want, click on    
CANCEL here.  
Just be aware that many DOS programs (like zippers) can      
not handle command lines longer than 127 characters.    
So don't specify too many individual filenames!    
/* Lines for Input the Date: input box: */
Type the date you wish to use in MMDDYY format.      
For example, type 120192 for Dec 1, 1992.     
(Leave this line blank to use today's date.)  
/* Lines for OPTIONAL Temporary directory: input box in Temp: subroutine: */
Type the full drive:\pathname of the directory you   
want Zip to use as a working directory, if any.   
For example, D:\TEMP or F:\DOWNLOAD\WORKING           
You should choose a directory on the fastest drive   
you have, provided that it has enough free space.   
If there's already a directory name there, and you  
don't want to use that one, just backspace over it.  
/* Lines for Temporary directory: input box in ViewExec: subroutine: */
Type the full drive:\pathname of the directory   
you want Unzip to use as a working directory.    
For example, D:\TEMP or F:\DOWNLOAD\WORKING           
/* Lines for Oops! message box in ViewExec: subroutine: */
Sorry, there's a problem here.  We can't create the temporary     
directory so we have to abort this operation in order to avoid     
the possibility of deleting any of your files.     
/* Lines for WAIT!  Warning! message box in ViewDiskFiles: subroutine: */
Move this little dialog box out of your way while you read your .INF
file, but DO NOT CLICK ON OK until you've CLOSED the VIEW program
(by double-clicking mouse button one on the little square in the top
left corner of the VIEW window).  If you ignore this piece of advice,
the Zip Engine program will try to delete its temporary files while
VIEW is still using one of them, and OS/2 won't allow that to happen
so you'll have temp files left behind on your hard drive to find two
months from now and wonder where they came from.
/* Lines for Edit the .INI File: message box in EditCfg: subroutine: */
Welcome to the ZIPENG.INI modifying routine.  In order for you to get
here, either you chose this option from the main menu or else your zip/
unzip programs could not be found.  Now let's see your .INI file.....
ZIPENG.INI file cannot be found in the current directory or on
the PATH, which are the only places The Zip Engine looks for it.
We'll have to create a new ZIPENG.INI file in the current directory
now.  If you want to move it later, you may.  By the way, your current
/* Lines for Exit Line: message box: */
The eighth possible line in the .INI file is "exit action" which tells
the Program what you want to do after the first zip/unzip function is
completed.  The default is to go back to this Program's main menu,
where you can perform more zip/unzip functions.  If you ever do more
than one zip/unzip function in a row, this saves you the time it takes
this Program to initialize itself.  But if you rarely want to perform
more than one function at a time, and you don't want to have to click
on CANCEL to exit this Program afterward, then choose EXIT below.  Then
this Program will just exit instead of returning you to the main menu
when your first zip/unzip function is completed.
/* Lines for Zipper Line: input box: */
The first possible line in the .INI file is "zipper", which tells   
the Program the full drive:\path\filename.ext of your zip program.     
Shown below is whatever is already assigned to your "zipper" line,       
if anything.  You may leave it as it is, change it to whatever you     
like, or leave it blank to remove the whole line from the .INI   
file.  But if you do that (remove the line), your zip program must     
be named PKZIP.EXE and must be on the PATH command in the           
OS/2 CONFIG.SYS file or the Zip Engine program will not work.       
/* Lines for Unzipper Line: input box: */
The second possible line in the .INI file is "unzipper", which tells    
the Program the full drive:\path\filename.ext of your unzip program.    
Shown below is whatever is already assigned to your "unzipper" line,      
if anything.  You may leave it as it is, change it to whatever you    
like, or leave it blank to remove the whole line from the .INI  
file.  But if you do that (remove the line), your unzip program must    
be named PKUNZIP.EXE and must be on the PATH command in the           
OS/2 CONFIG.SYS file or the Zip Engine program will not work.      
/* Lines for Directory Lines: multbox: */
The next three possible lines in the .INI file are "temporary directory",
"extract directory", and "default directory".  The input box shows what
you now have assigned to those lines, if anything.  You may leave them as
they are, change them to whatever you want, or leave them blank to remove
them from the .INI file altogether.  The Program will not allow you to
specify "\" (root directory of current drive), however---that could cause
all sorts of problems if you ever happen to use The Zip Engine with a 
floppy drive as your current drive.  So if you want to specify the root
directory, include the drive letter along with it, as in "C:\" or "D:\".
     The "temporary directory" line tells Zip Engine where to create its
own temp files, and what to use whenever you unzip files with the "-b"
switch (though the Program does allow you to override that setting for
the "-b" switch if you so desire).
     The "extract directory" line tells Zip Engine what directory name
to supply for your approval each time you see the Target Directory
Name: input box.
     The "default directory" line tells Zip Engine what directory to show
you in the FileBox dialog boxes and to supply for your approval each time
you see the .ZIP File to Create: input box.  If you always keep all of
your .ZIP files in a certain directory, then you probably want to put
its name here.  If you leave this line blank, Zip Engine will use the
current directory for those functions.
/* Lines for Default options information window: */
The last three possible lines in the .INI file are 'zip options', 'unzip
options', and 'global options', which tell the Program to use certain
switches on your command lines without you telling it each time to do so.
Do NOT put anything here until you are very familiar with both your
zipper's syntax AND this Program.  What you put here for zip options
will go on the command line during any procedure that takes you to the
'Zip Special OPTIONAL Switches' dialog box.  What you put here for un-
zip options will go on the command line during any procedure that takes
you to the 'Unzip Special OPTIONAL Switches' dialog box.  What you put
here for global options will go on the command line during any proced-
ure that uses your Zip/Unzip program in any way, even things that happen
in the background such as the unzipping of a .ZIP file into a temporary
directory in order to let you view text files within the .ZIP file.  Be
careful to use only options that you want used EVERY time Zip Engine
performs ANY sort of Zip/Unzip operation.  Remember that's only for the
global options.  The zip options and unzip options are more forgiving,
because they only apply to the commands which you have a chance to edit
before executing.  So for example, if you like to use the -$ switch with
zip ALMOST all the time, you can put it here and then at times when you
don't want to use it, you can just edit it out of the finished command
line before executing it.  DO NOT choose an option from the Special OP-
TIONAL Switches dialog boxes that has already been specified here, be-
cause it'll already BE on the command line.  Note that there is no error
checking in this routine, so don't put anything into these default options
lines that you're not SURE is valid and appropriate.  And DO NOT put any-
thing into the zip options and unzip options lines that is also in your
global options line, or it will show up twice on the command line.
----------
Click on the "Options lines:" dialog box above when ready to fill it in.
/* Lines for About..... message box: */
                       The Zip Engine.
     A graphical approach to zipping and unzipping files.
   Copyright 1992-93 by Kari Jackson and Bart Toulouse.
                          Version 1.0
        3201 Monroe Street, Omaha NE  68107-4048
   Click on OK to view the Zip Engine program's .DOC file
       or click on CANCEL to return to the main menu.
/* Lines for Sorry! message box in About..... routine: */
The ZIPENG.DOC file cannot be found either in the current directory
or an any of the directories on your CONFIG.SYS file's PATH
command.  Returning you to the main menu......
/* Lines for Sorry! message box in PAR2.q = '-s' routine: */
You did not specify any files to add to this
.ZIP file.  You can only add password protec-
tion to the files that are being added.
Therefore, you cannot use the -s switch at
this time.
/* Lines for Sorry! message box in CreateError: subroutine: */
We were not able to create the temporary file which is
needed in order to complete this procedure.  The most
common reason for this to occur is a full root direc-
tory.  The root directory of a hard drive can contain
no more than 512 entries (or 112 for a double-density
floppy, or 224 for a high-density floppy).  If you're
already at the limit, then Zip Engine cannot create
another.  If the root directory size limit is not the
problem, then you should run CHKDSK and see what it
finds.  Returning you to the main menu now.
/* Lines for VSAY window in ViewOnly: subroutine: */
When the Unzip program's window comes up, click on it to bring it to the
foreground, and type your password (or if you clicked on OK by mistake,
just type anything and hit Enter).  You will not see the Unzip program
ask you for input, since its output is being redirected into a temp file.
/* Lines for Window Action window: */
The ninth possible line in the .INI file is "start switch" which tells
the Program how you want your zip/unzip windows to behave.  By default,
the windows are opened by the START command with the /C switch, which
means they close by themselves as soon as the zip/unzip command has
finished.  This means you don't have to close them yourself, but it
also means it's not terribly easy for you to see the output of the zip/
unzip command if you want to.  If you often want to see that output,
you can tell Zip Engine to START those windows with the /K switch
instead, which means they'll stay open so you can see the output, and
then you'll have to type the EXIT command in those windows yourself to
close them when you're done reading them.
/* Lines for EditColors: subroutine: */
The last lines in the .INI file specify the colors you
want to use for Zip Engine windows.  That is, not dialog
boxes---those use the Menu Text and Dialog Background
colors from whatever color scheme is current on your
Desktop.  This is about the windows that open up for
information purposes only, with no mouse buttons on them,
like this one.  The default window colors are white on
blue.  This Edit-the-.INI-File routine will not use
your chosen colors, but the rest of The Zip Engine will.

* *** WARNING *** WARNING *** WARNING *** WARNING *** WARNING *** WARNING *** *
* This file must NOT be modified in ANY way.  If you even open it with a text *
* editor and save it without changing anything, it could be ruined.  Because  *
* most text editors do not preserve spaces at the end of lines.  This file's  *
* end-of-line spaces ARE NECESSARY.  If you've opened this file to look at it,*
* get out, go away, close the file without SAVING it!  Now!                   *
* *** WARNING *** WARNING *** WARNING *** WARNING *** WARNING *** WARNING *** *
